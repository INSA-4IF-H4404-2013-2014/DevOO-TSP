\newglossaryentry{superviseur}
{
	name={superviseur},
	description={Employé supervisant la bonne conduite des livraisons},
	sort={superviseur}
}

\newglossaryentry{livraison}
{
	name={livraison (Effectuer une) / \textit{delivery}},
	description={Acte d'effectuer un trajet dont l'aboutissement est la livraison d'un colis et le remplissage d'un reçu, ou bien une notification de non-livraison},
	sort={livraison}
}

\newglossaryentry{client}
{
	name={client},
	description={Personne morale ou physique qui envoie ou reçoit des articles via la société de transport},
	sort={client}
}

\newglossaryentry{itineraire}
{
	name={itinéraire / \textit{itinerary}},
	description={Ensemble ordonné de tronçons à suivre afin d'effectuer une tournée. Commence et finit par un nœud étant soit l'adresse de l'entrepôt, soit l'adresse d'une livraison. Soit $I_{t_k}$ le $k^\text{ème}$ nœud du tronçon t appartenant à l'itinéraire $I$, alors $I_{{(t+1)}_{1}} = I_{{t}_{card(t)}}$ est vérifiée},
	sort={itineraire}
}

\newglossaryentry{rue}
{
	name={rue / \textit{street}},
	description={Ensemble de tronçons ayant un nom de rue identique. Celle-ci relie plusieurs nœuds},
	sort={rue}
}

\newglossaryentry{troncon}
{
	name={tronçon / \textit{arc}},
	description={Portion unidirectionnelle d'une rue reliant deux nœuds},
	sort={troncon}
}

\newglossaryentry{noeud}
{
	name={nœud / \textit{node}},
	description={Point de jonction duquel arrivent ou partent un ou plusieurs tronçons. Adresse d'un entrepôt ou d'une livraison},
	sort={noeud}
}

\newglossaryentry{reseau}
{
	name={réseau / \textit{network}},
	description={Ensemble de nœuds reliés par des tronçons. Un réseau est assimilable à une ville},
	sort={reseau}
}

\newglossaryentry{entrepot}
{
	name={entrepôt / \textit{warehouse}},
	description={Lieu de départ d'une tournée, lié à la journée en cours (l'entrepôt peut changer d'une journée à l'autre)},
	sort={entrepot}
}

\newglossaryentry{journee}
{
	name={journée},
	description={Intervalle de temps séparant le lever du soleil à son coucher},
	sort={journee}
}

\newglossaryentry{plageHoraire}
{
	name={plage horaire} / \textit{time frame},
	description={Durée, sous-ensemble d'une journée, définie par une heure de début et de fin},
	sort={plage}
}

\newglossaryentry{tournee}
{
	name={tournée / \textit{round}},
	description={ensemble ordonné et réparti dans le temps des itinéraires à parcourir dans une journée de travail par un livreur dans le but d'effectuer des livraisons},
	sort={tournee}
}

\newglossaryentry{livreur}
{
	name={livreur},
	description={Employé de l'entreprise disposant d'un véhicule et assigné à une ou plusieurs tournées dans une journée de travail dont la mission est d'en effectuer les livraisons},
	sort={livreur}
}
